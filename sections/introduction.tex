\section{Introduction}

  In this report I will expand on the work done previously for the company known as ACME. A system has been designed for the company to 
  move into a digital and online world. This system was designed around the following business goals:
  \begin{itemize}
    \item Increase profits/customers
    \item Improve documentation resilience and navigability
    \item Cater to the student demographic
    \item Automate/speed up time intensive tasks
  \end{itemize}

  This new system was designed to move ACME away from a paper-based system, which was slow to update and had little to no \textit{'backup-ability'},
  as well as from using phone/email to communicate. In addition to this they wanted to take advantage of the student population. Research in 
  previous work concluded that adding cryptocurrency payments could be a good way to engage this younger audience and could provide an interesting 
  angle for marketing. The new software system would move all the old functionality into a software-based approach, these features include:

  \begin{itemize}
    \item Adding customer information
    \item Taking payment
    \item Handling the return of a vehicle
    \item Starting a new hire
    \item Adding a new car to the system
  \end{itemize}

  A web-based application is to be created to support the above features. Where users can book and manage accounts and rentals without the need of a member
  of staff. This web application will handle the payments, customer details, car details and rental details using databases and third-party providers that 
  are discussed in the previous work.

  This report will continue this work by looking at the project in terms of architectural design, any security and safety concerns that could arise from the 
  project and how the software will handle faults and promote availability. I have included the sequence diagrams from the previous work in 
  \hyperref[sec:AppendixA]{\textbf{Appendix A}} to help understand the flow of the new systems.

\newpage